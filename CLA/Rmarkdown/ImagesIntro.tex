% Options for packages loaded elsewhere
\PassOptionsToPackage{unicode}{hyperref}
\PassOptionsToPackage{hyphens}{url}
%
\documentclass[
]{article}
\usepackage{amsmath,amssymb}
\usepackage{lmodern}
\usepackage{iftex}
\ifPDFTeX
  \usepackage[T1]{fontenc}
  \usepackage[utf8]{inputenc}
  \usepackage{textcomp} % provide euro and other symbols
\else % if luatex or xetex
  \usepackage{unicode-math}
  \defaultfontfeatures{Scale=MatchLowercase}
  \defaultfontfeatures[\rmfamily]{Ligatures=TeX,Scale=1}
\fi
% Use upquote if available, for straight quotes in verbatim environments
\IfFileExists{upquote.sty}{\usepackage{upquote}}{}
\IfFileExists{microtype.sty}{% use microtype if available
  \usepackage[]{microtype}
  \UseMicrotypeSet[protrusion]{basicmath} % disable protrusion for tt fonts
}{}
\makeatletter
\@ifundefined{KOMAClassName}{% if non-KOMA class
  \IfFileExists{parskip.sty}{%
    \usepackage{parskip}
  }{% else
    \setlength{\parindent}{0pt}
    \setlength{\parskip}{6pt plus 2pt minus 1pt}}
}{% if KOMA class
  \KOMAoptions{parskip=half}}
\makeatother
\usepackage{xcolor}
\usepackage[margin=1in]{geometry}
\usepackage{color}
\usepackage{fancyvrb}
\newcommand{\VerbBar}{|}
\newcommand{\VERB}{\Verb[commandchars=\\\{\}]}
\DefineVerbatimEnvironment{Highlighting}{Verbatim}{commandchars=\\\{\}}
% Add ',fontsize=\small' for more characters per line
\usepackage{framed}
\definecolor{shadecolor}{RGB}{248,248,248}
\newenvironment{Shaded}{\begin{snugshade}}{\end{snugshade}}
\newcommand{\AlertTok}[1]{\textcolor[rgb]{0.94,0.16,0.16}{#1}}
\newcommand{\AnnotationTok}[1]{\textcolor[rgb]{0.56,0.35,0.01}{\textbf{\textit{#1}}}}
\newcommand{\AttributeTok}[1]{\textcolor[rgb]{0.77,0.63,0.00}{#1}}
\newcommand{\BaseNTok}[1]{\textcolor[rgb]{0.00,0.00,0.81}{#1}}
\newcommand{\BuiltInTok}[1]{#1}
\newcommand{\CharTok}[1]{\textcolor[rgb]{0.31,0.60,0.02}{#1}}
\newcommand{\CommentTok}[1]{\textcolor[rgb]{0.56,0.35,0.01}{\textit{#1}}}
\newcommand{\CommentVarTok}[1]{\textcolor[rgb]{0.56,0.35,0.01}{\textbf{\textit{#1}}}}
\newcommand{\ConstantTok}[1]{\textcolor[rgb]{0.00,0.00,0.00}{#1}}
\newcommand{\ControlFlowTok}[1]{\textcolor[rgb]{0.13,0.29,0.53}{\textbf{#1}}}
\newcommand{\DataTypeTok}[1]{\textcolor[rgb]{0.13,0.29,0.53}{#1}}
\newcommand{\DecValTok}[1]{\textcolor[rgb]{0.00,0.00,0.81}{#1}}
\newcommand{\DocumentationTok}[1]{\textcolor[rgb]{0.56,0.35,0.01}{\textbf{\textit{#1}}}}
\newcommand{\ErrorTok}[1]{\textcolor[rgb]{0.64,0.00,0.00}{\textbf{#1}}}
\newcommand{\ExtensionTok}[1]{#1}
\newcommand{\FloatTok}[1]{\textcolor[rgb]{0.00,0.00,0.81}{#1}}
\newcommand{\FunctionTok}[1]{\textcolor[rgb]{0.00,0.00,0.00}{#1}}
\newcommand{\ImportTok}[1]{#1}
\newcommand{\InformationTok}[1]{\textcolor[rgb]{0.56,0.35,0.01}{\textbf{\textit{#1}}}}
\newcommand{\KeywordTok}[1]{\textcolor[rgb]{0.13,0.29,0.53}{\textbf{#1}}}
\newcommand{\NormalTok}[1]{#1}
\newcommand{\OperatorTok}[1]{\textcolor[rgb]{0.81,0.36,0.00}{\textbf{#1}}}
\newcommand{\OtherTok}[1]{\textcolor[rgb]{0.56,0.35,0.01}{#1}}
\newcommand{\PreprocessorTok}[1]{\textcolor[rgb]{0.56,0.35,0.01}{\textit{#1}}}
\newcommand{\RegionMarkerTok}[1]{#1}
\newcommand{\SpecialCharTok}[1]{\textcolor[rgb]{0.00,0.00,0.00}{#1}}
\newcommand{\SpecialStringTok}[1]{\textcolor[rgb]{0.31,0.60,0.02}{#1}}
\newcommand{\StringTok}[1]{\textcolor[rgb]{0.31,0.60,0.02}{#1}}
\newcommand{\VariableTok}[1]{\textcolor[rgb]{0.00,0.00,0.00}{#1}}
\newcommand{\VerbatimStringTok}[1]{\textcolor[rgb]{0.31,0.60,0.02}{#1}}
\newcommand{\WarningTok}[1]{\textcolor[rgb]{0.56,0.35,0.01}{\textbf{\textit{#1}}}}
\usepackage{graphicx}
\makeatletter
\def\maxwidth{\ifdim\Gin@nat@width>\linewidth\linewidth\else\Gin@nat@width\fi}
\def\maxheight{\ifdim\Gin@nat@height>\textheight\textheight\else\Gin@nat@height\fi}
\makeatother
% Scale images if necessary, so that they will not overflow the page
% margins by default, and it is still possible to overwrite the defaults
% using explicit options in \includegraphics[width, height, ...]{}
\setkeys{Gin}{width=\maxwidth,height=\maxheight,keepaspectratio}
% Set default figure placement to htbp
\makeatletter
\def\fps@figure{htbp}
\makeatother
\setlength{\emergencystretch}{3em} % prevent overfull lines
\providecommand{\tightlist}{%
  \setlength{\itemsep}{0pt}\setlength{\parskip}{0pt}}
\setcounter{secnumdepth}{-\maxdimen} % remove section numbering
\ifLuaTeX
  \usepackage{selnolig}  % disable illegal ligatures
\fi
\IfFileExists{bookmark.sty}{\usepackage{bookmark}}{\usepackage{hyperref}}
\IfFileExists{xurl.sty}{\usepackage{xurl}}{} % add URL line breaks if available
\urlstyle{same} % disable monospaced font for URLs
\hypersetup{
  pdftitle={Images Introduction},
  pdfauthor={Will Mitchell},
  hidelinks,
  pdfcreator={LaTeX via pandoc}}

\title{Images Introduction}
\author{Will Mitchell}
\date{2023-01-20}

\begin{document}
\maketitle

{} Download

\hypertarget{introduction-to-creating-vectors-matrices-and-images-in-r}{%
\subsection{Introduction to creating vectors, matrices and images in
R}\label{introduction-to-creating-vectors-matrices-and-images-in-r}}

\hypertarget{install-packages}{%
\subsubsection{Install packages}\label{install-packages}}

You'll need the ``magick'' package installed. Use the package manager in
RStudio (on the right side of the screen by default) or type
\emph{install.packages(`magick')} in the Console.

\hypertarget{learn-to-create-vectors}{%
\subsubsection{Learn to create vectors}\label{learn-to-create-vectors}}

One of the most useful features in R is this colon trick for creating a
list (or \emph{vector}) of consecutive integers:

\begin{Shaded}
\begin{Highlighting}[]
\NormalTok{v }\OtherTok{\textless{}{-}} \DecValTok{4}\SpecialCharTok{:}\DecValTok{9}
\FunctionTok{print}\NormalTok{(v)}
\end{Highlighting}
\end{Shaded}

\begin{verbatim}
## [1] 4 5 6 7 8 9
\end{verbatim}

This can be used in many interesting ways. Once you have the vector, you
can do arithmetic on it, like this:

\begin{Shaded}
\begin{Highlighting}[]
\NormalTok{w }\OtherTok{\textless{}{-}}\NormalTok{ v }\SpecialCharTok{/} \DecValTok{10} \SpecialCharTok{+} \DecValTok{5}
\FunctionTok{print}\NormalTok{(w)}
\end{Highlighting}
\end{Shaded}

\begin{verbatim}
## [1] 5.4 5.5 5.6 5.7 5.8 5.9
\end{verbatim}

An alternative way to create the same vector is to type the entries
individually and collect them with the important \emph{c} command:

\begin{Shaded}
\begin{Highlighting}[]
\NormalTok{w }\OtherTok{\textless{}{-}} \FunctionTok{c}\NormalTok{(}\FloatTok{5.4}\NormalTok{,}\FloatTok{5.5}\NormalTok{,}\FloatTok{5.6}\NormalTok{,}\FloatTok{5.7}\NormalTok{,}\FloatTok{5.8}\NormalTok{,}\FloatTok{5.9}\NormalTok{)}
\FunctionTok{print}\NormalTok{(w)}
\end{Highlighting}
\end{Shaded}

\begin{verbatim}
## [1] 5.4 5.5 5.6 5.7 5.8 5.9
\end{verbatim}

\textbf{Warning}: at some point you will be tempted to create a variable
with the name \emph{c}, the third lower case letter. Do not do it!!!!
That overwrites the important ``collect'' function. The same goes for
lower case \emph{t}, which is the ``transpose'' function.

If you have two vectors of the same length, you can make a plot!

\begin{Shaded}
\begin{Highlighting}[]
\NormalTok{x }\OtherTok{\textless{}{-}}\NormalTok{ (}\DecValTok{0}\SpecialCharTok{:}\DecValTok{50}\NormalTok{)  }\SpecialCharTok{*}\NormalTok{ pi }\SpecialCharTok{/} \DecValTok{50} \CommentTok{\# numbers from zero to pi  }
\NormalTok{y }\OtherTok{\textless{}{-}} \FunctionTok{sin}\NormalTok{(x)  }\CommentTok{\# create some data }
\NormalTok{y[}\DecValTok{20}\NormalTok{] }\OtherTok{\textless{}{-}} \FloatTok{0.2} \CommentTok{\# put an outlier in the 20th spot}
\FunctionTok{plot}\NormalTok{(x,y, }\AttributeTok{main=}\StringTok{"A simple plot"}\NormalTok{)}
\end{Highlighting}
\end{Shaded}

\includegraphics{ImagesIntro_files/figure-latex/unnamed-chunk-5-1.pdf}

\hypertarget{exercises-on-vectors}{%
\subsubsection{Exercises on vectors}\label{exercises-on-vectors}}

\begin{itemize}
\tightlist
\item
  Add the command \emph{print(x)} to this cell to see another use of the
  colon syntax: changing many entries at once.
\end{itemize}

\begin{Shaded}
\begin{Highlighting}[]
\NormalTok{x }\OtherTok{\textless{}{-}} \DecValTok{0}\SpecialCharTok{*}\NormalTok{(}\DecValTok{1}\SpecialCharTok{:}\DecValTok{6}\NormalTok{)}
\NormalTok{x[}\DecValTok{3}\SpecialCharTok{:}\DecValTok{5}\NormalTok{] }\OtherTok{\textless{}{-}} \DecValTok{2}
\FunctionTok{print}\NormalTok{(x)}
\end{Highlighting}
\end{Shaded}

\begin{verbatim}
## [1] 0 0 2 2 2 0
\end{verbatim}

\begin{itemize}
\item
  Use the colon syntax to create a list of 101 evenly spaced numbers,
  starting at -1 and ending at 3.
\item
  Create the vector \((2,2,2,2,5,6,7,8,9,-1)\).
\end{itemize}

\begin{Shaded}
\begin{Highlighting}[]
\NormalTok{evenly\_space }\OtherTok{\textless{}{-}} \FunctionTok{seq}\NormalTok{(}\SpecialCharTok{{-}}\DecValTok{1}\NormalTok{, }\DecValTok{3}\NormalTok{, }\AttributeTok{length.out=}\DecValTok{101}\NormalTok{)}
\NormalTok{vct }\OtherTok{\textless{}{-}} \FunctionTok{c}\NormalTok{(}\DecValTok{2}\NormalTok{, }\DecValTok{2}\NormalTok{, }\DecValTok{2}\NormalTok{, }\DecValTok{2}\NormalTok{, }\DecValTok{5}\NormalTok{, }\DecValTok{6}\NormalTok{, }\DecValTok{7}\NormalTok{, }\DecValTok{8}\NormalTok{, }\DecValTok{9}\NormalTok{, }\SpecialCharTok{{-}}\DecValTok{1}\NormalTok{)}
\FunctionTok{print}\NormalTok{(evenly\_space)}
\end{Highlighting}
\end{Shaded}

\begin{verbatim}
##   [1] -1.00 -0.96 -0.92 -0.88 -0.84 -0.80 -0.76 -0.72 -0.68 -0.64 -0.60 -0.56
##  [13] -0.52 -0.48 -0.44 -0.40 -0.36 -0.32 -0.28 -0.24 -0.20 -0.16 -0.12 -0.08
##  [25] -0.04  0.00  0.04  0.08  0.12  0.16  0.20  0.24  0.28  0.32  0.36  0.40
##  [37]  0.44  0.48  0.52  0.56  0.60  0.64  0.68  0.72  0.76  0.80  0.84  0.88
##  [49]  0.92  0.96  1.00  1.04  1.08  1.12  1.16  1.20  1.24  1.28  1.32  1.36
##  [61]  1.40  1.44  1.48  1.52  1.56  1.60  1.64  1.68  1.72  1.76  1.80  1.84
##  [73]  1.88  1.92  1.96  2.00  2.04  2.08  2.12  2.16  2.20  2.24  2.28  2.32
##  [85]  2.36  2.40  2.44  2.48  2.52  2.56  2.60  2.64  2.68  2.72  2.76  2.80
##  [97]  2.84  2.88  2.92  2.96  3.00
\end{verbatim}

\begin{Shaded}
\begin{Highlighting}[]
\FunctionTok{print}\NormalTok{(vct)}
\end{Highlighting}
\end{Shaded}

\begin{verbatim}
##  [1]  2  2  2  2  5  6  7  8  9 -1
\end{verbatim}

\hypertarget{learn-to-create-matrices}{%
\subsubsection{Learn to create
matrices}\label{learn-to-create-matrices}}

Suppose we want to build this matrix: \[ A = \begin{pmatrix}
0.2&0.8&0.8&1.0\\
0.6&0.6&1.0&0.0\\
0.2&0.0&0.0&0.5
\end{pmatrix}\]

We can do this by creating a vector with all of the entries, and then
making it 2D by giving it the desired dimensions. Note that we start
with all of the elements in the first column, then the second, and so
on.

\begin{Shaded}
\begin{Highlighting}[]
\NormalTok{A }\OtherTok{\textless{}{-}} \FunctionTok{c}\NormalTok{(}\FloatTok{0.2}\NormalTok{,}\FloatTok{0.6}\NormalTok{,}\FloatTok{0.2}\NormalTok{,}\FloatTok{0.8}\NormalTok{,}\FloatTok{0.6}\NormalTok{,}\FloatTok{0.0}\NormalTok{,}\FloatTok{0.8}\NormalTok{,}\FloatTok{1.0}\NormalTok{,}\FloatTok{0.0}\NormalTok{,}\FloatTok{1.0}\NormalTok{,}\FloatTok{0.0}\NormalTok{,}\FloatTok{0.5}\NormalTok{)}
\FunctionTok{dim}\NormalTok{(A) }\OtherTok{\textless{}{-}} \FunctionTok{c}\NormalTok{(}\DecValTok{3}\NormalTok{,}\DecValTok{4}\NormalTok{)}
\FunctionTok{print}\NormalTok{(A)}
\end{Highlighting}
\end{Shaded}

\begin{verbatim}
##      [,1] [,2] [,3] [,4]
## [1,]  0.2  0.8  0.8  1.0
## [2,]  0.6  0.6  1.0  0.0
## [3,]  0.2  0.0  0.0  0.5
\end{verbatim}

It's kind of annoying to type every entry in the matrix, though. For
matrices with more structure (such as large regions with identical
entries) there are shortcuts. Here's an example:

\[ B = \begin{pmatrix}0&0&0&0&0&0&0&0&0&0&0\\
0&0&0&0&0&0&0&0&2&0&0\\
0&1&1&1&1&1&0&0&0&0&0\\
0&1&1&1&1&1&0&0&0&4&4\\
0&1&1&1&1&1&0&0&0&4&4\\
0&0&0&0&0&0&0&0&0&4&4\end{pmatrix}\]

To create \(B\), we'll start with a matrix containing all zeros. Then
we'll try to change whole blocks of entries simultaneously:

\begin{Shaded}
\begin{Highlighting}[]
\NormalTok{B }\OtherTok{\textless{}{-}} \FunctionTok{matrix}\NormalTok{(}\DecValTok{0}\NormalTok{,}\DecValTok{6}\NormalTok{,}\DecValTok{11}\NormalTok{)}
\NormalTok{B[}\DecValTok{3}\SpecialCharTok{:}\DecValTok{5}\NormalTok{,}\DecValTok{2}\SpecialCharTok{:}\DecValTok{6}\NormalTok{] }\OtherTok{\textless{}{-}} \DecValTok{1}
\NormalTok{B[}\DecValTok{2}\NormalTok{,}\DecValTok{9}\NormalTok{] }\OtherTok{\textless{}{-}} \DecValTok{2}
\NormalTok{B[}\DecValTok{4}\SpecialCharTok{:}\DecValTok{6}\NormalTok{,}\DecValTok{10}\SpecialCharTok{:}\DecValTok{11}\NormalTok{] }\OtherTok{\textless{}{-}} \DecValTok{4}
\FunctionTok{print}\NormalTok{(B)}
\end{Highlighting}
\end{Shaded}

\begin{verbatim}
##      [,1] [,2] [,3] [,4] [,5] [,6] [,7] [,8] [,9] [,10] [,11]
## [1,]    0    0    0    0    0    0    0    0    0     0     0
## [2,]    0    0    0    0    0    0    0    0    2     0     0
## [3,]    0    1    1    1    1    1    0    0    0     0     0
## [4,]    0    1    1    1    1    1    0    0    0     4     4
## [5,]    0    1    1    1    1    1    0    0    0     4     4
## [6,]    0    0    0    0    0    0    0    0    0     4     4
\end{verbatim}

It can also useful be useful to create matrices with randomly chosen
entries. Here are two versions, first with \emph{uniformly} distributed
entries (between 0 and 1) and next with \emph{normally} distributed
entries (with mean zero and standard deviation 1). In both cases we
start with a vector and then use the \emph{matrix} command to reshape it
into a\ldots{} matrix with given dimensions.

\begin{Shaded}
\begin{Highlighting}[]
\NormalTok{r1 }\OtherTok{\textless{}{-}} \FunctionTok{matrix}\NormalTok{(}\FunctionTok{runif}\NormalTok{(}\DecValTok{20}\NormalTok{),}\DecValTok{4}\NormalTok{,}\DecValTok{5}\NormalTok{)}
\FunctionTok{print}\NormalTok{(r1)}
\end{Highlighting}
\end{Shaded}

\begin{verbatim}
##            [,1]      [,2]      [,3]       [,4]      [,5]
## [1,] 0.47466999 0.7174521 0.4645567 0.04514971 0.3190841
## [2,] 0.08692093 0.1417700 0.6820196 0.59121181 0.5001712
## [3,] 0.24576708 0.7486554 0.5824023 0.65655525 0.5139146
## [4,] 0.99892624 0.1657658 0.7108239 0.95512667 0.8139455
\end{verbatim}

\begin{Shaded}
\begin{Highlighting}[]
\NormalTok{r2 }\OtherTok{\textless{}{-}} \FunctionTok{matrix}\NormalTok{(}\FunctionTok{rnorm}\NormalTok{(}\DecValTok{12}\NormalTok{),}\DecValTok{4}\NormalTok{,}\DecValTok{3}\NormalTok{)}
\FunctionTok{print}\NormalTok{(r2)}
\end{Highlighting}
\end{Shaded}

\begin{verbatim}
##            [,1]       [,2]      [,3]
## [1,] -1.0209595 -0.8340193 0.6535734
## [2,]  0.3143791 -0.4044857 0.4717157
## [3,] -0.5941875 -0.7191404 0.4123738
## [4,]  0.2353453 -0.9129796 0.8680073
\end{verbatim}

One more trick: diagonal matrices! Check this out:

\begin{Shaded}
\begin{Highlighting}[]
\NormalTok{vec }\OtherTok{\textless{}{-}} \FunctionTok{c}\NormalTok{(}\DecValTok{4}\NormalTok{,}\DecValTok{5}\NormalTok{,}\DecValTok{6}\NormalTok{,}\DecValTok{7}\NormalTok{,}\DecValTok{8}\NormalTok{,}\DecValTok{9}\NormalTok{)}
\NormalTok{C }\OtherTok{=} \FunctionTok{diag}\NormalTok{(vec)}
\FunctionTok{print}\NormalTok{(C)}
\end{Highlighting}
\end{Shaded}

\begin{verbatim}
##      [,1] [,2] [,3] [,4] [,5] [,6]
## [1,]    4    0    0    0    0    0
## [2,]    0    5    0    0    0    0
## [3,]    0    0    6    0    0    0
## [4,]    0    0    0    7    0    0
## [5,]    0    0    0    0    8    0
## [6,]    0    0    0    0    0    9
\end{verbatim}

\hypertarget{exercises-create-these-matrices}{%
\subsubsection{Exercises: create these
matrices}\label{exercises-create-these-matrices}}

\[
W = \begin{pmatrix}
16&-2&\pi\\0&0&0\\1&-1&12\\0&0&1
\end{pmatrix}
\]

\begin{Shaded}
\begin{Highlighting}[]
\CommentTok{\# your code here: end with something like "print(W)"}
\end{Highlighting}
\end{Shaded}

\[
X = \begin{pmatrix}
2&2&2&2\\
2&2&2&1\\
2&2&1&1\\
2&1&1&1\\
1&1&1&1\\
\end{pmatrix}
\]

\[
Y = \begin{pmatrix}
4&3&3&2&2&2&2&2&2\\
3&4&3&2&2&2&2&2&2\\
3&3&4&2&2&2&2&2&2\\
0&0&0&4&3&3&2&2&2\\
0&0&0&3&4&3&2&2&2\\
0&0&0&3&3&4&2&2&2\\
0&0&0&0&0&0&4&3&3\\
0&0&0&0&0&0&3&4&3\\
0&0&0&0&0&0&3&3&4\\
\end{pmatrix}
\]

\hypertarget{create-images-from-matrices}{%
\subsubsection{Create images from
matrices}\label{create-images-from-matrices}}

Now let's create some images! To make a color image, we'll create three
matrices (one for each of the Red, Green and Blue \emph{color
channels}). Here are the three matrices for our first example:

\begin{Shaded}
\begin{Highlighting}[]
\NormalTok{myRed }\OtherTok{\textless{}{-}} \FunctionTok{matrix}\NormalTok{(}\DecValTok{0}\NormalTok{,}\DecValTok{500}\NormalTok{,}\DecValTok{500}\NormalTok{) }\CommentTok{\# a 500x500 grid of zeros}
\NormalTok{myRed[}\DecValTok{100}\SpecialCharTok{:}\DecValTok{300}\NormalTok{,}\DecValTok{100}\SpecialCharTok{:}\DecValTok{300}\NormalTok{] }\OtherTok{\textless{}{-}} \DecValTok{1} \CommentTok{\# change some entries into ones}

\NormalTok{myGreen }\OtherTok{\textless{}{-}} \FunctionTok{matrix}\NormalTok{(}\DecValTok{0}\NormalTok{,}\DecValTok{500}\NormalTok{,}\DecValTok{500}\NormalTok{)}
\NormalTok{myGreen[}\DecValTok{200}\SpecialCharTok{:}\DecValTok{400}\NormalTok{,}\DecValTok{150}\SpecialCharTok{:}\DecValTok{350}\NormalTok{] }\OtherTok{\textless{}{-}} \DecValTok{1}

\NormalTok{myBlue }\OtherTok{\textless{}{-}} \FunctionTok{matrix}\NormalTok{(}\DecValTok{0}\NormalTok{,}\DecValTok{500}\NormalTok{,}\DecValTok{500}\NormalTok{)}
\NormalTok{myBlue[}\DecValTok{150}\SpecialCharTok{:}\DecValTok{250}\NormalTok{,}\DecValTok{200}\SpecialCharTok{:}\DecValTok{500}\NormalTok{] }\OtherTok{\textless{}{-}} \DecValTok{1}
\end{Highlighting}
\end{Shaded}

We could use any values \textbf{between zero and one} in these matrices.

Next, we combine them into a 3D array:

\begin{Shaded}
\begin{Highlighting}[]
\NormalTok{data }\OtherTok{\textless{}{-}} \FunctionTok{array}\NormalTok{(}\FunctionTok{c}\NormalTok{(myRed,myGreen,myBlue),}\AttributeTok{dim =} \FunctionTok{c}\NormalTok{(}\DecValTok{500}\NormalTok{,}\DecValTok{500}\NormalTok{,}\DecValTok{3}\NormalTok{))}
\end{Highlighting}
\end{Shaded}

We load the library `magick', which we previously installed:

\begin{Shaded}
\begin{Highlighting}[]
\FunctionTok{library}\NormalTok{(}\StringTok{\textquotesingle{}magick\textquotesingle{}}\NormalTok{)}
\end{Highlighting}
\end{Shaded}

\begin{verbatim}
## Linking to ImageMagick 6.9.12.3
## Enabled features: cairo, fontconfig, freetype, heic, lcms, pango, raw, rsvg, webp
## Disabled features: fftw, ghostscript, x11
\end{verbatim}

Then we can display the image and also save it in our working directory:

\begin{Shaded}
\begin{Highlighting}[]
\NormalTok{im }\OtherTok{\textless{}{-}} \FunctionTok{image\_read}\NormalTok{(data) }
\FunctionTok{print}\NormalTok{(im) }\CommentTok{\# this displays the image in RStudio and knitted documents}
\end{Highlighting}
\end{Shaded}

\begin{verbatim}
##   format width height colorspace matte filesize density
## 1    PNG   500    500       sRGB FALSE        0   72x72
\end{verbatim}

\includegraphics[width=6.94in]{ImagesIntro_files/figure-latex/unnamed-chunk-16-1}

\begin{Shaded}
\begin{Highlighting}[]
\FunctionTok{image\_write}\NormalTok{(im,}\StringTok{"mycolors.png"}\NormalTok{) }\CommentTok{\# this saves the image as a separate file}
\end{Highlighting}
\end{Shaded}

\hypertarget{create-an-image-showing-the-first-letter-in-your-name}{%
\subsubsection{Create an image showing the first letter in your
name}\label{create-an-image-showing-the-first-letter-in-your-name}}

\begin{Shaded}
\begin{Highlighting}[]
\NormalTok{myRed }\OtherTok{\textless{}{-}} \FunctionTok{matrix}\NormalTok{(}\DecValTok{0}\NormalTok{,}\DecValTok{500}\NormalTok{,}\DecValTok{500}\NormalTok{) }\CommentTok{\# a 500x500 grid of zeros}
\NormalTok{myRed[}\DecValTok{0}\SpecialCharTok{:}\DecValTok{500}\NormalTok{,}\DecValTok{0}\SpecialCharTok{:}\DecValTok{500}\NormalTok{] }\OtherTok{\textless{}{-}} \DecValTok{1} \CommentTok{\# change some entries into ones}

\NormalTok{myGreen }\OtherTok{\textless{}{-}} \FunctionTok{matrix}\NormalTok{(}\DecValTok{0}\NormalTok{,}\DecValTok{500}\NormalTok{,}\DecValTok{500}\NormalTok{)}
\NormalTok{shiftleft }\OtherTok{\textless{}{-}} \DecValTok{175}
\NormalTok{shiftDown }\OtherTok{\textless{}{-}} \DecValTok{50}
\ControlFlowTok{for}\NormalTok{ (x }\ControlFlowTok{in} \DecValTok{0} \SpecialCharTok{:} \DecValTok{100}\NormalTok{) \{}
  \ControlFlowTok{for}\NormalTok{ (t }\ControlFlowTok{in} \DecValTok{0} \SpecialCharTok{:} \DecValTok{50}\NormalTok{) \{}
\NormalTok{    t }\OtherTok{\textless{}{-}}\NormalTok{ t }\SpecialCharTok{{-}}\NormalTok{ shiftleft}
\NormalTok{    myGreen[x}\SpecialCharTok{+}\NormalTok{shiftDown, }\DecValTok{400}\SpecialCharTok{+}\NormalTok{x}\SpecialCharTok{+}\NormalTok{t] }\OtherTok{\textless{}{-}} \DecValTok{1}
\NormalTok{    myGreen[x}\SpecialCharTok{+}\NormalTok{shiftDown, }\DecValTok{400}\SpecialCharTok{{-}}\NormalTok{x}\SpecialCharTok{+}\NormalTok{t] }\OtherTok{\textless{}{-}} \DecValTok{1}
\NormalTok{    myGreen[}\DecValTok{100}\SpecialCharTok{+}\NormalTok{x}\SpecialCharTok{+}\NormalTok{shiftDown, }\DecValTok{300}\SpecialCharTok{+}\NormalTok{x}\SpecialCharTok{+}\NormalTok{t] }\OtherTok{\textless{}{-}} \DecValTok{1}
\NormalTok{    myGreen[}\DecValTok{195}\SpecialCharTok{+}\NormalTok{x}\SpecialCharTok{+}\NormalTok{shiftDown, }\DecValTok{395}\SpecialCharTok{+}\NormalTok{x}\SpecialCharTok{+}\NormalTok{t] }\OtherTok{\textless{}{-}} \DecValTok{1}
\NormalTok{    myGreen[}\DecValTok{395}\SpecialCharTok{{-}}\NormalTok{x}\SpecialCharTok{+}\NormalTok{shiftDown, }\DecValTok{395}\SpecialCharTok{+}\NormalTok{x}\SpecialCharTok{+}\NormalTok{t] }\OtherTok{\textless{}{-}} \DecValTok{1}
\NormalTok{    myGreen[}\DecValTok{395}\SpecialCharTok{{-}}\NormalTok{x}\SpecialCharTok{+}\NormalTok{shiftDown, }\DecValTok{395}\SpecialCharTok{{-}}\NormalTok{x}\SpecialCharTok{+}\NormalTok{t] }\OtherTok{\textless{}{-}} \DecValTok{1}
\NormalTok{  \}}

\NormalTok{\}}

\NormalTok{myBlue }\OtherTok{\textless{}{-}} \FunctionTok{matrix}\NormalTok{(}\DecValTok{0}\NormalTok{,}\DecValTok{500}\NormalTok{,}\DecValTok{500}\NormalTok{)}


\NormalTok{data }\OtherTok{\textless{}{-}} \FunctionTok{array}\NormalTok{(}\FunctionTok{c}\NormalTok{(myRed,myGreen,myBlue),}\AttributeTok{dim =} \FunctionTok{c}\NormalTok{(}\DecValTok{500}\NormalTok{,}\DecValTok{500}\NormalTok{,}\DecValTok{3}\NormalTok{))}

\NormalTok{im }\OtherTok{\textless{}{-}} \FunctionTok{image\_read}\NormalTok{(data) }
\FunctionTok{print}\NormalTok{(im) }\CommentTok{\# this displays the image in RStudio and knitted documents}
\end{Highlighting}
\end{Shaded}

\begin{verbatim}
##   format width height colorspace matte filesize density
## 1    PNG   500    500       sRGB FALSE        0   72x72
\end{verbatim}

\includegraphics[width=6.94in]{ImagesIntro_files/figure-latex/unnamed-chunk-17-1}

\begin{Shaded}
\begin{Highlighting}[]
\FunctionTok{image\_write}\NormalTok{(im,}\StringTok{"mycolors.png"}\NormalTok{)}
\end{Highlighting}
\end{Shaded}

When you're done, save it in
\href{https://docs.google.com/document/d/1s1axnlolpOGDqnVTa38m4AzHkJXJrK690bzXXAyIJVI/edit?usp=sharing}{this
Google doc}. I'll print them (in color) and they can become name cards
for our class.

\hypertarget{create-an-image-showing-random-noise}{%
\subsubsection{Create an image showing random
noise}\label{create-an-image-showing-random-noise}}

Is \emph{runif} or \emph{rnorm} more useful?

\hypertarget{find-out-what-happens-if-you-use-numbers-larger-than-1-or-smaller-than-0-in-the-three-color-channel-matrices}{%
\subsubsection{Find out what happens if you use numbers larger than 1 or
smaller than 0 in the three color channel
matrices}\label{find-out-what-happens-if-you-use-numbers-larger-than-1-or-smaller-than-0-in-the-three-color-channel-matrices}}

\hypertarget{create-an-image-that-looks-gray}{%
\subsubsection{Create an image that looks
gray}\label{create-an-image-that-looks-gray}}

You can do this by using exactly the same 2D matrix for all three color
channels.

\hypertarget{add-some-random-noise-to-the-image-we-already-made}{%
\subsubsection{Add some random noise to the image we already
made}\label{add-some-random-noise-to-the-image-we-already-made}}

\hypertarget{load-an-external-image-and-convert-to-matrix-form}{%
\subsubsection{Load an external image and convert to matrix
form}\label{load-an-external-image-and-convert-to-matrix-form}}

\begin{Shaded}
\begin{Highlighting}[]
\CommentTok{\# before uncommenting, save an image to your working directory! }
\CommentTok{\# this assumes the image is called "trees.png"}

\CommentTok{\#a = image\_read("trees.png")}
\CommentTok{\#A = as.integer(a[[1]])}
\CommentTok{\#dim(A)}
\CommentTok{\#print(a)}
\end{Highlighting}
\end{Shaded}

\hypertarget{important-example-a-rank-one-matrix}{%
\subsubsection{Important example: a rank-one
matrix}\label{important-example-a-rank-one-matrix}}

Here is an example of a \emph{rank-one} matrix, meaning it contains all
possible products of entries in two vectors. Check that you understand
how the entries in \(A\) come from the entries of \(a_1\) and \(a_2\)
here:

\begin{Shaded}
\begin{Highlighting}[]
\NormalTok{a1 }\OtherTok{\textless{}{-}} \FunctionTok{c}\NormalTok{(}\DecValTok{4}\NormalTok{,}\DecValTok{5}\NormalTok{,}\DecValTok{6}\NormalTok{,}\DecValTok{7}\NormalTok{)}
\FunctionTok{print}\NormalTok{(a1)}
\end{Highlighting}
\end{Shaded}

\begin{verbatim}
## [1] 4 5 6 7
\end{verbatim}

\begin{Shaded}
\begin{Highlighting}[]
\NormalTok{a2 }\OtherTok{\textless{}{-}} \FunctionTok{c}\NormalTok{(}\DecValTok{1}\NormalTok{,}\DecValTok{0}\NormalTok{,}\SpecialCharTok{{-}}\DecValTok{2}\NormalTok{)}
\FunctionTok{print}\NormalTok{(a2)}
\end{Highlighting}
\end{Shaded}

\begin{verbatim}
## [1]  1  0 -2
\end{verbatim}

\begin{Shaded}
\begin{Highlighting}[]
\NormalTok{A }\OtherTok{\textless{}{-}}\NormalTok{ a1 }\SpecialCharTok{\%*\%} \FunctionTok{t}\NormalTok{(a2)}
\FunctionTok{print}\NormalTok{(A)}
\end{Highlighting}
\end{Shaded}

\begin{verbatim}
##      [,1] [,2] [,3]
## [1,]    4    0   -8
## [2,]    5    0  -10
## [3,]    6    0  -12
## [4,]    7    0  -14
\end{verbatim}

Rank-one matrices are interesting because they contain lots of numbers,
but not a lot of information (just the information contained in a pair
of vectors). A \emph{low-rank} matrix can be written as a sum of a small
number of rank-one matrices. If you have a large data set, it is
extremely wonderful if you can find a low-rank approximation for it. We
will explore some methods for doing this later in the course.

Here is a much larger example of a rank-one matrix turned into an image:

\begin{Shaded}
\begin{Highlighting}[]
\CommentTok{\# create an interesting vector: like a sine wave }
\NormalTok{t }\OtherTok{=}\NormalTok{ (}\DecValTok{0}\SpecialCharTok{:}\DecValTok{500}\NormalTok{) }\SpecialCharTok{/} \DecValTok{500} \CommentTok{\# this is a vector of 501 numbers ranging from 0 to 1}
\NormalTok{x }\OtherTok{=} \FunctionTok{sin}\NormalTok{(}\DecValTok{3}\SpecialCharTok{*}\NormalTok{pi}\SpecialCharTok{*}\NormalTok{t}\SpecialCharTok{\^{}}\DecValTok{2}\NormalTok{)}\SpecialCharTok{\^{}}\DecValTok{2} \SpecialCharTok{*}\NormalTok{ (}\FloatTok{0.85+0.15}\SpecialCharTok{*}\FunctionTok{runif}\NormalTok{(}\DecValTok{501}\NormalTok{)) }
\NormalTok{y }\OtherTok{=} \FunctionTok{sin}\NormalTok{(}\DecValTok{5}\SpecialCharTok{*}\NormalTok{pi}\SpecialCharTok{*}\NormalTok{t)}\SpecialCharTok{\^{}}\DecValTok{2} \SpecialCharTok{*}\NormalTok{ (}\FloatTok{0.85+0.15}\SpecialCharTok{*}\FunctionTok{runif}\NormalTok{(}\DecValTok{501}\NormalTok{))   }
\NormalTok{A }\OtherTok{=}\NormalTok{ x }\SpecialCharTok{\%*\%} \FunctionTok{t}\NormalTok{(y) }\CommentTok{\# creates a rank{-}one matrix from two vectors}
\NormalTok{data }\OtherTok{=} \FunctionTok{array}\NormalTok{(}\FunctionTok{c}\NormalTok{(A,A,A),}\FunctionTok{c}\NormalTok{(}\DecValTok{501}\NormalTok{,}\DecValTok{501}\NormalTok{,}\DecValTok{3}\NormalTok{))}
\FunctionTok{print}\NormalTok{(}\FunctionTok{image\_read}\NormalTok{(data)) }\CommentTok{\# this displays the image in RStudio and knitted documents}
\end{Highlighting}
\end{Shaded}

\begin{verbatim}
##   format width height colorspace matte filesize density
## 1    PNG   501    501       sRGB FALSE        0   72x72
\end{verbatim}

\includegraphics[width=6.96in]{ImagesIntro_files/figure-latex/unnamed-chunk-20-1}

Do you see how this image is related to the graphs of the vectors
\emph{x} and \emph{y}?

\begin{Shaded}
\begin{Highlighting}[]
\FunctionTok{plot}\NormalTok{(x)}
\end{Highlighting}
\end{Shaded}

\includegraphics{ImagesIntro_files/figure-latex/unnamed-chunk-21-1.pdf}

\begin{Shaded}
\begin{Highlighting}[]
\FunctionTok{plot}\NormalTok{(y)}
\end{Highlighting}
\end{Shaded}

\includegraphics{ImagesIntro_files/figure-latex/unnamed-chunk-22-1.pdf}

\end{document}
